\documentclass[dvipdfmx,11pt]{beamer}		% for my notebook computer and my Mac computer
%\documentclass[11pt]{beamer}			% for overleaf

\usepackage{amsmath}
\usepackage{amssymb}
%\usepackage{amsthm}
\usepackage{multicol}
\usepackage{listings}
\usepackage{otf}
\usepackage{algorithm}
\usepackage{algorithmic}
\usepackage{tikz}
\usepackage{mathtools}
\usepackage{comment}
\usetheme{Berlin}	%全体のデザイン

\useoutertheme[subsection=false]{smoothbars}	%デザインのカスタマイズ

\setbeamertemplate{navigation symbols}{}	%右下のちっちゃいナビゲーション記号を非表示

\AtBeginSubsection[]	%サブセクションごとに目次を表示
{\begin{frame}{Contents}
    \begin{multicols}{2}
        \tableofcontents[currentsubsection]
    \end{multicols}
\end{frame}}
\newtheorem{defi}{Definition}
\newtheorem{thm}[defi]{Theorem}
\newtheorem{prop}[defi]{Proposition}
\newtheorem{conj}[defi]{Conjecture}
\newtheorem{exam}[defi]{Example}
\newtheorem{prob}[defi]{Problem}
\newtheorem{set}[defi]{Setting}
\newtheorem{claim}[defi]{Claim}

\newcommand{\N}{\mathbb{N}}
\newcommand{\R}{\mathbb{R}}
\newcommand{\X}{\mathcal{X}}
\newcommand{\Y}{\mathcal{Y}}
\newcommand{\Hil}{\mathcal{H}}
\newcommand{\Loss}{\mathcal{L}_{D}}
\newcommand{\MLsp}{(\X, \Y, D, \Hil, \Loss)}

\newcommand{\argmax}{\mathop{\rm arg~max}\limits}
\newcommand{\argmin}{\mathop{\rm arg~min}\limits}


\title{Generative Adversarial Networks}
\author{Takaya KOIZUMI}
\institute{Mathematical Science, B4}
\date{Applied Mathematics and Physics informal seminor}
\begin{document}
    \begin{frame}\frametitle{}
        \titlepage
    \end{frame}
    \section*{Contents}
    \begin{frame}\frametitle{Contents}
        \begin{multicols}{2}
            \tableofcontents
        \end{multicols}
    \end{frame}
    \section*{References}
    \begin{frame}\frametitle{References}
        \begin{thebibliography}{9}
            \beamertemplatetextbibitems
            \bibitem{UAT} G. Cybenko, Approximation by Superpositions of a Sigmoidal Function, 
                            Mathmatics of control, signal and systems, vol. 2, no. 4, 1989
		    \bibitem{PFN} Preferred Networks, ディープラーニング入門 Chainer チュートリアル,
            https://tutorials.chainer.org/ja/index.html, 2019
	    \end{thebibliography}
    \end{frame}
\end{document}


