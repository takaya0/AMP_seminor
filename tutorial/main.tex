\documentclass[dvipdfmx,11pt]{beamer}		% for my notebook computer and my Mac computer
%\documentclass[11pt]{beamer}			% for overleaf

\usepackage{amsmath}
\usepackage{amssymb}
%\usepackage{amsthm}

\usetheme{Berlin}	%全体のデザイン

\useoutertheme[subsection=false]{smoothbars}	%デザインのカスタマイズ

\setbeamertemplate{navigation symbols}{}	%右下のちっちゃいナビゲーション記号を非表示

\AtBeginSubsection[]	%サブセクションごとに目次を表示
{\begin{frame}{Contents}
	\tableofcontents[currentsubsection]
\end{frame}}

\newtheorem{defi}{Definition}
\newtheorem{thm}[defi]{Theorem}
\newtheorem{prop}[defi]{Proposition}
\newtheorem{conj}[defi]{Conjecture}
\newtheorem{prob}[defi]{Problem}
\newtheorem{set}[defi]{Setting}
\newtheorem{claim}[defi]{Claim}

\newcommand{\R}{\mathbb{R}}
\newcommand{\X}{\mathcal{X}}
\newcommand{\Y}{\mathcal{Y}}
\newcommand{\Hil}{\mathcal{H}}
\title{Machine Learning tutorial}
\author{Takaya KOIZUMI}
\institute{Mathematical Science, B4}
\date{Applied Mathematics and Physics informal seminor}
\begin{document}
    \begin{frame}\frametitle{}
        \titlepage
    \end{frame}
    \section*{Contents}
    \begin{frame}\frametitle{Contents}
        \tableofcontents
    \end{frame}
    \section{機械学習の枠組み}
    \subsection{機械学習とは}
    \begin{frame}\frametitle{機械学習とは}
        機械学習とは, 「関数近似論」である.
        \begin{block}{世の中で機械学習を使って実現したと言われている技術}
            \begin{enumerate}
                \item 翻訳~~($\{$全ての日本語$\}$$\to$$\{$全ての英語$\}$という関数)
                \item メール分類~~($\{$全てのメールの文章$\}$$\to$$\{$迷惑メール, 非迷惑メール$\}$という関数)
                \item 音声認識~~($\{$音声$\}$$\to$$\{$文章$\}$という関数)
            \end{enumerate}
        \end{block}
        もちろん, 間違いを起こすこともある. (大事なメールが, 迷惑メールに入ることも...)
    \end{frame}
    \begin{frame}\frametitle{数学的には}
        前スライドの話を集合論を用いて, もう少し数学的にきちんと書くならば, 以下のようになるだろう.
        \begin{block}{機械学習?}
            $\X$, $\Y$をそれぞれ$\R^n$, $\R^m$の部分集合とする. 
            この時, 良い関数$f:\X\to\Y$を見つけることを機械学習という. 
        \end{block} 
        しかし, この定義には以下の問題がある.
        \begin{block}{上の定義の問題点}
            \begin{enumerate}
                \item 候補となる関数が多すぎる. (ヒントも何もないのに探せない)
                \item 良い関数とは何か, 定義されていない.
            \end{enumerate}
        \end{block}
    \end{frame}
    \subsection{機械学習の数学的定式化へ}
    \begin{frame}\frametitle{前半の問題解消}
        では, まず前半の「候補となる関数が多すぎる.」という問題を解決していこう. \\ \indent
        この問題の解決方法として, 人間がヒント(条件)を与えてあげることで, 関数全ての
        集合ではなく, ある程度絞った集合$\Hil$にするということを考える. この$\Hil$のことを仮設空間(Hyposesis space)と呼ぶ.
        \begin{defi}[仮設空間]
            $\X$, $\Y$をそれぞれ$\R^n$, $\R^m$の部分集合とする. この時, 集合
            \begin{equation*}
                \Hil :=\{f_{\theta} :\X\to\Y\mid\text{$f_{\theta}$に関する条件}\}
            \end{equation*}
            のことを仮設空間と呼び, $\X$を特徴量空間, $\Y$をラベル空間と呼ぶ.
        \end{defi}
    \end{frame}
    \begin{frame}\frametitle{後半の問題解消}
        では, 後半の「良い関数」というものを定義していこう. 
        機械学習において, 良い関数とは, 未知のデータ$X$に対して正しい値$Y$を返す関数である. 
        そのために, 関数$f$に対してその良さを表す指標である汎化誤差を定義する.
        \begin{defi}[汎化誤差, 損失関数]
            $\Hil$を仮設空間, $(\Omega, \mathcal{F}, \mathbb{P})$を確率空間, $\rho$をデータの確率分布とする.
            この時, 汎化誤差$\ell:\Hil\to\R$を, 
            \begin{equation*}
                \ell(f_{\theta}) = \mathbb{E}_{(X, Y)\sim\rho}[l(f_{\theta}(X), Y)]
            \end{equation*}
            と定義する. ここで, $l:\Y\times\Y\to\R$は損失関数と呼ばれる凸関数である. 
        \end{defi}
    \end{frame}
    \begin{frame}
        \frametitle{損失関数の具体例}
        \begin{block}{損失関数}
            ここで, よく使われる損失関数の例をいくつか述べておく. 
            \begin{enumerate}
                \item 2乗損失関数 $l(y_1, y_2) = (y_1 - y_2)^2$
                \item 交差エントロピー誤差 $l(y_1, y_2) = -y_1\log y_2$
            \end{enumerate}
        \end{block}
        これで, 「良い関数」を作るためには, 汎化誤差$\ell$を最小化させるような仮設空間$\Hil$の元$f$を見つけば良いと言うことになったわけだが, 
        汎化誤差には期待値が含まれるため, 直接最適化させることが難しい. そのため, 持っているデータを利用して別の関数を用意し, その関数を
        最小化することを考える. 
    \end{frame}
    \begin{frame}
        \frametitle{データと経験損失関数}
        \begin{defi}[データ]
            $(\Omega, \mathcal{F}, \mathbb{P})$を確率空間, $\rho$をデータの確率分布とする.
            $\{(X_n, Y_n)\}_{n = 1}^{N}$を$\rho$に従う独立な確率変数列とする. $\{(X_n, Y_n)\}_{n = 1}^{N}$の観測値$\{(X_n(\omega), Y_n(\omega))\}_{n = 1}^{N}$
            のことをデータ(Data)と呼び, $D = \{(x_n, y_n)\}_{n = 1}^{N}$と表記する.
        \end{defi}
        \begin{defi}[経験損失関数]
            $\Hil$を仮設空間, $D = \{(x_n, y_n)\}_{n = 1}^{N}$をデータ, $l:\Y\times\Y\to\R$を
            損失関数とする. この時, 経験損失関数$L_{D}:\Hil\to\R$を,
            \begin{equation*}
                L_{D}(f_{\theta}) = \sum_{n = 1}^{N}l(f_{\theta}(x_n), y_n)
            \end{equation*}
            と定義する. 
        \end{defi}
        
    
    \end{frame}
    \section{単回帰と重回帰}
    \subsection{単回帰分析}
    \begin{frame}\frametitle{最も基礎的なモデル}
        test
    \end{frame}
    
    \subsection{重回帰分析}
    \subsection{数値実験}
    \section{過学習と正則化}
    \begin{frame}\frametitle{非線形データへの対応}
        
    \end{frame}
\end{document}


